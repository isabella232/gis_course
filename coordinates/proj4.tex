
\begin{frame}
    \frametitle{PROJ4}
    PROJ4 библиотека для работы с системами координат (используется в GRASS GIS,  MapServer,  PostGIS,  Thuban,  OGDI,  Mapnik,  TopoCad, and  OGRCoordinateTransformation и др.).

    База описаний проекций EPSG (\url{http://www.epsg.org/}) хранит данные в формате PROJ4.
\end{frame}

\begin{frame}[fragile]
    \frametitle{Описание проекций в формате proj4}
    Некоторые параметры:
    \begin{itemize}
        \item \verb!+proj! Название проекции
        \item \verb!+lat_0! Широта начала координат
        \item \verb!+lon_0! Осевой меридиан
        \item \verb!+k! Масштабный коэффициент
        \item \verb!+x! Восточное смещение
        \item \verb!+ellps! Эллипсоид
        \item \verb!+towgs84! Параметры перехода к WGS84
        \item \verb!+units! Единицы измерения
    \end{itemize}


\end{frame}

\begin{frame}[fragile]
    \frametitle{Описание проекций в формате proj4}
    Пример: описание системы координат Гаусса-Крюгера 9-й зоны на эллипсоиде Крассовского:
    \begin{verbatim}
    +proj=tmerc +lat_0=0 +lon_0=51 +k=1
    +x_0=9500000 +y_0=0
    +ellps=krass
    +towgs84=23.92,-141.27,-80.9,-0,0.35,0.82,-0.12
    +units=m
    \end{verbatim}

    Пример: широта/долгота на WGS84:
    \begin{verbatim}
    +proj=longlat +datum=WGS84
    \end{verbatim}
\end{frame}

\begin{frame}[fragile]
    \frametitle{Примеры преобразований}
    Программа cs2cs производит преобразование координат
    заданными в указанных системах координат.
    \begin{verbatim}
    cs2cs +proj=latlong +datum=NAD83
                   +to +proj=utm +zone=10 +datum=NAD27 -r <<EOF
             45d15'33.1"   111.5W
             45d15.551666667N   -111d30
             +45.25919444444    111d30'000w
             EOF
    \end{verbatim}
    Будет произведено преобразование входных координат из широта/долгота на NAD83
     в 10-ю зону системы UTM на NAD27.
\end{frame}

