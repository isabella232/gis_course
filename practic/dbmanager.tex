
\begin{frame}[allowframebreaks]
    \frametitle{Задачи}
    \begin{itemize}
        \item Создать выборку районов где численность городского населения превышает сельское (разница населения)
        \item Создать слой населенных пунктов принадлежащих районам из предыдущего задания (связь по полям Tanp.Agnum=ag200.Kodname)
        \item Создать слой населенных пунктов принадлежащих районам из предыдущего задания (связь по геометрии)
        \item Определить численность населения в районах созданных в 1930,1938,1965 годах.
        \item Построить запрос для отображения района с минимальным приростом. Использовать слой Ag200.
        \item Построить картографический слой административных районов РТ, пересекаемых хотя бы одной улучшенной шоссейной дорогой.
        \item Построить картографический слой административных районов РТ, пересекаемых ровно четырьмя шоссейными дорогами.
        \item Построить картографический слой улучшенных шоссейных дорог, целиком находящихся внутри одного административного района РТ.
        \item Выбрать все дороги, которые пересекаются (граничат) с aдминистративным центором. (Административные центры хранятся в точечном слое taadmp, поэтому для поиска полигональных административных центров нужно предварительно выбрать объекты из слоя Tanp).
        \item Построить последовательность запросов для поиска населенных пунктов (картографический слой Tanp), расположенных на расстоянии не более 10 км от шоссейных дорог в тех административных районах, плотность населения которых выше средней по республике.
    \end{itemize}
\end{frame}

\begin{frame}
    \frametitle{Упрощение геометрии}
    CREATE VIEW simpl AS SELECT pk, name, Simplify(geom, 200) FROM ag200 WHERE name='Верхнеуслонский'
\end{frame}
