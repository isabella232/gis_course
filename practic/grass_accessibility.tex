
\begin{frame}
    \frametitle{Построение зон доступности с использованием GRASS}
    \begin{block}{Задача}
        Имеется информация о точках и маршрутах движения. Требуется построить зоны доступности от каждой точки, например,
        зоны пешеходной доступности в 200, 400, 600, \dots метров.

        Ограничения: пространство считается неоднородным с точки зрения движения. Например, пешеход пойдет скорее по дорогам, чем
        по бездорожью; существуют участки, которые недоступны для прохода (здания, промзоны и т.п.).
    \end{block}

\end{frame}

\begin{frame}
    \frametitle{Построение зон доступности с использованием GRASS}
    Задача может быть решена с двух разных подходов:
    \begin{itemize}
        \item Векторный подход (построение графа дорог и поиск пути по этому графу):
        \begin{itemize}
            \item преимущество подхода: точность расчетов движения по дорогам;
            \item недостатки подхода: невозможно построить маршрут через области, в которых не проходит дорога (в реальной жизни пешеход может срезать углы).
        \end{itemize}
        \item Растровый подход (построение растра стоимости движения и расчет суммарной стоимости движения от точек):
        \begin{itemize}
            \item преимущество подхода: при создании растра стоимости движения мы не ограничены местоположениями дорог и можем строить пути в произвольных областях;
            \item недостатки подхода: при создании растра стоимости нужно назначить цену движения через ячейку растра; выбор объективной стоимости может оказаться достаточно сложной задачей.
        \end{itemize}
    \end{itemize}

\end{frame}

