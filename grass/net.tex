
%~ \begin{frame}[fragile]
    %~ \frametitle{Инструменты GRASS для работы с сетями}
    %~ \begin{itemize}
        %~ \item v.net: создание сети.
        %~ \item v.net.iso: изодистанции
        %~ \item v.net.path кратчайший путь
    %~ \end{itemize}
%~ \end{frame}

% Конвертируем полигоны в линии
% v.type roads out=roads1 type=boundary,line,centroid,point
% v.clean input=roads1@PERMANENT type=line tool=snap,break,rmdupl,rmsa thresh=20,0,0,0 output=test --o
%  v.category test opt=report
%  v.edit test type=point tool=delete cat=1-400

% Считаем изолинии:
% v.net streets points=schools out=streets_net op=connect thresh=200
% v.net.iso in=streets_net out=streets_net_iso ccats=1-1000000 nlayer=2 costs=200,400,600,800
% v.category streets_net_iso opt=report # 5 cats

% Считаем кратчайший путь между двумя узлами:
% 1 - ID, растояние считается от узла категории 15 до узла категории 21
% номер категории можно узнать, если включить отображение "Display category numbers" в менеджере слоев и показать категории со слоя 2 (v.net op=connect складывает присоединенные узлы во второй слой)
% echo "1 15 21" | v.net.path streets_net out=spath


% Linear reference system (LRS)
% v.lrs.create создание системы
% v.lrs.label to create stationing on LRS
% v.lrs.segment to create points/segments from LRS
% v.lrs.where to find the line ID



