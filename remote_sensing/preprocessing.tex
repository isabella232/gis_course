
\begin{frame}
    \frametitle{Радиометрическая коррекция}
    Данные, которые приходят со спутника измеряются <<в попугаях>> ($DN$), поскольку они сжаты в определенный диапазон значений.

    Для того, чтобы получить реальное зарегистрирванное сенсором излучение ($L_j$), нужно рассчитать его по формуле:

    \begin{equation*}
        L_j = gain_j DN_j + bias_j
    \end{equation*}
    где параметры $gain_j$ и $bias_j$ указываются в метаданных.

    На этом часто останавливаются (особенно, если анализируется один снимок). Если нужно учесть атмосферные эффекты (анализ разновременных снимков), выполняют атмосферную коррекцию.
\end{frame}

\begin{frame}[allowframebreaks]
    \frametitle{Атмосферная коррекция}
    Технологию можно прочитать подробнее в Шовенгердт Р.А. Дистанционное зондирование. Модели и методы обработки изображений. М.: Техносфера, 2010. - 560 с. - ISBN: 978-5-94836-244-1; (перевод книги Robert A. Schowengerdt Models and Methods for Image Processing ISBN: 978-0-12-369407-2)

    Полная атмосферная коррекция выполняется редко, поскольку для этой процедуры требуется знать состояние атмосферы на всей площади снимка, а это сложно. Поэтому состояние атмосферы оценивают по самим снимкам и полученные оценки используют для коррекции. Более-менее стандартная схема работы такова:

Обозначения: $b$ --- номер канала; $DN_b$ --- значения пикселей снимков в попугаях; зарегистрированное датчиком излучение для канала $b$ обозначим $L^s_b$.
\begin{enumerate}
    \item Преобразование из попугаев в излучение производится линейно:
    $$L^s_b = gain_bDN_b +bias_b$$
    параметры преобразования берутся из метаданных снимка.
    \item Собственно атмосферная коррекция. Излучение у поверхности Земли ($L_b$) в точке $(x,y)$ при предположении, что отраженным от Земли рассеяным излучением можно пренебречь, рассчитывается:
    $$L_b(x,y) = \frac{L^s_b(x,y) - L_b^{sp}}{t_{vb}},$$
    где $t_{vb}$ --- спектральный коэффициент пропускания атмосферы (табличное значение, зависит от длины волны), а $L^{sp}_b$ --- плотность потока рассеянного излучения (не дошедшего до Земли и вернувшегося на датчик). Для сцен с однородным рельефом и датчиков, снимающих в надире обычно считается постоянной на всей площади снимка (но зависит от длины волны).
    \item Оценка плотности потока рассеянного излучения $L^{sp}_b$. Способов много разной степени сложности. Простой но действенный --- найти черный объект(ы) и вычесть из всех пикселей значения яркости пикселей, соответствующих черному объекту(ам)
    \item Поправка на угол восхождения Солнца и расчет отражательной способности:
    $$\rho_b(x,y) = \frac{\pi L_b(x,y)} {t_{sb}E^0_b\cos(\Theta(x,y))},$$
    где $E^0_b$ --- спектральная плотность энергетической освещенности на верней границе атмосферы (хорошо известная величина), $\Theta$ --- угол направления на Солнце.

    Как считать излучение у поверхности Земли показано на шаге 2, спектральная плотность потока на верхней границы атмосферы --- табличная величина, угол падения излучения (зависит от угла восхождения Солнца и рельефа). Остается коэффициент пропускания атмосферы $t_{sb}$, который как-то нужно оценить (в идеале должен измеряться в момент съемки).
\end{enumerate}

\end{frame}
