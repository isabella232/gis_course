
\begin{frame}[fragile]
    \frametitle{Базовый запрос на языке SQL}
    \begin{verbatim}
    SELECT поля
    FROM таблицы
    WHERE условие
    \end{verbatim}
\end{frame}

\begin{frame}[allowframebreaks]
    \frametitle{Пространственные расширения языка SQL: базовые операции}
    Базовые операции, применимые ко всем геометрическим типам данных. Например, SpatialReference возвращает базовую систему координат, в которой описана геометрия объекта. К числу распространенных систем координат относятся широко известная система широт и долгот, а также часто используемая система Universal Traversal Mercator (UTM).
    \begin{itemize}
        \item SpatialReference(), ST\_SRID(), SRID(): Возвращает базовую систему координат геометрии
        \item Envelope(), ST\_Envelope(): Возвращает минимальный ортогональный ограничивающий прямоугольник геометрии
        \item Export(), AsText(), ST\_AsText(): Возвращает альтернативное представление геометрии
        \item IsEmpty(), St\_IsEmplty: Возвращает истинное значение, если геометрия является пустым множеством
        \item IsSimple(), ST\_IsSimple(): Возвращает истинное значение, если геометрия является простой (без самопересечений)
        \item Boundary(), ST\_Boundary(): Возвращает границы геометрии (у полигона --- linestring)
        \item Simplyfy(g, tolerance): Возвращает упрощенную геометрию с заданной погрешностью.
    \end{itemize}
\end{frame}


\begin{frame}[allowframebreaks]
    \frametitle{Топологические операции и операции над множествами}
    \begin{itemize}
        \item Equals(g1, g2): Возвращает истинное значение, если внутренние области и
        границы обеих геометрий пространственно равны
        \item Disjoint(g1, g2): Возвращает истинное значение, если границы и внутренняя область g1 и g2 не пересекаются
        \item Intersects(g1, g2): Возвращает истинное значение, если геометрии имеют общие элементы
        \item Touches(g1, g2): Возвращает истинное значение, если границы двух поверхностей пересекаются, а внутренние области --- нет
        \item Crosses: Возвращает истинное значение, если внутренняя область поверхности пересекается кривой
        \item Within(g1, g2): Возвращает истинное значение, если внутренняя область одной геометрии не пересекается с внешней областью другой геометрии g1 в g2
        \item Contains(g1, g2): Проверяет, содержит ли одна геометрия другую g2 в g1.
        \item Overlaps: Возвращает истину, если внутренние области двух геометрий имеют непустое пересечение
    \end{itemize}
\end{frame}

\begin{frame}[allowframebreaks]
    \frametitle{Пространственый анализ}
    \begin{itemize}
        \item Distance: Возвращает кратчайшее расстояние между двумя геометриями
        \item Buffer(geom, R): Возвращает геометрию, содержащую все точки, лежащие на указанном или меньшем расстоянии от данной геометрии
        \item ConvexHull: Возвращает наименьшее выпуклое геометрическое множество, заключающее в себе данную геометрию
        \item Intersection: Возвращает геометрическое пересечение двух геометрий
        \item Difference(g1, g2): Возвращает фрагмент геометрии, который не пересекается с другой геометрией g1 - g2
        \item SymmDiff(g1, g2): Возвращает фрагменты двух геометрий, которые не пересекаются друг с другом
    \end{itemize}
\end{frame}









