
\begin{frame}[allowframebreaks]
    \frametitle{Анализ}
    \begin{itemize}
        \item Матрица расстояний. Измеряет расстояние между точками двух точечных слоёв и выдает результат в виде (a) квадратной матрицы расстояний, (b) линейной матрицы расстояний, или (c) суммы расстояний. Можно ограничить расчет только для k ближайших точек.
        \item Сумма расстояний в полигонах. Рассчитывает сумму расстояний для линий линейного слоя в пределах каждого полигона другого (векторного полигонального) слоя.
        \item Количество точек в полигонах. Рассчитывает число точек точечного слоя, которые находятся в пределах каждого полигона другого (векторного полигонального) слоя.
        \item Список уникальных значений. Отображает список всех уникальных значений для указанного поля атрибутивной таблицы исходного векторного слоя.
        \item Базовая статистика. Рассчитывает основные статистики (среднее, стандартное отклонение, количество, сумму, коэффициент вариации) для указанного поля.
        \item Анализ близости. Рассчитывает значение близости для оценки степени сгруппированности точек в пределах точечного векторного слоя.

        Наблюдаемое среднее:
        \begin{equation*}
            D_o = \frac{\sum_{i=1}^n d_i}{n},
        \end{equation*}
        где $n$ --- число точек, $d_i$ --- расстояние от точки номер $i$ до ее ближайшего соседа.


        Ожидаемое среднее:
        \begin{equation*}
            D_e = \frac{0.5}{\sqrt{n/A}},
        \end{equation*}
        где $A$ --- площадь изучаемое территории.

        Индекс ближайших соседей:
        \begin{equation*}
            R = \frac{D_o}{D_e}
        \end{equation*}

        Z-показатель:
        \begin{equation*}
            z = \frac{D_o-D_e}{s_e},
        \end{equation*}
        где $S_e = \frac{0.26136}{\sqrt{n^2/A}}$

        \item Средние координаты. Рассчитывает среднеарифметические или средневзвешенные координаты центра для целого векторного слоя или для набора объектов, выбранного на основе уникальные значения из указанного поля.
        \item Пересечение линий. Рассчитывает местонахождения пересечений линий, создавая точечный шейп-файл с точками пересечений.
    \end{itemize}
\end{frame}


\begin{frame}[allowframebreaks]
    \frametitle{Выборка}
    \begin{itemize}
        \item Случайная выборка и Выборка в подмножествах: случайный выбор объектов из таблицы.
        \item Случайные точки. Cоздает псевдо-случайные точки в пределах границ указанного слоя..
        \item Регулярные точки. Создаёт регулярную сетку точек в пределах указаной области и экспортирует их в точечный шейп-файл..
        \item Регулярная сетка. Создаёт линейную или полигональную сетку, основываясь на заданном пользователем интервале..
        \item Пространственная выборка. Выделяет объекты на основе их положения относительно другого слоя, создавая новую выборку или добавляя/отнимая к/от текущей выборки. (Вместо этого инструмента лучше использовать <<Модули/Пространственный запрос>>)
        \item Полигон из границ слоя. Создаёт полигональный слой с единственным прямоугольным полигоном в соответствии с границами исходного растрового или векторного слоя.
    \end{itemize}
\end{frame}


\begin{frame}[allowframebreaks]
    \frametitle{Геообработка}
    \begin{itemize}
        \item Выпуклые оболочки. Создает минимально возможные выпуклые оболочки, или выпуклые оболочки на основе указанного поля.
        \item Буферные зоны. Создает буферные зоны вокруг объектов заданного пользователем размера, или используя размер из значений указанного поля.
        \item Пересечение. Совмещает слои таким образом, что в выходном слое содержатся только участки, в которых оба слоя пересекаются. Аттрибутивная таблица составляется из полей обоих слоев.
        \item Разность. Совмещает слои таким образом, что в выходном слое содержатся только те участки, которые не пересекаются со слоем отсечения.
        \item Cимметрическая разность. Совмещает слои таким образом, что в выходном слое содержатся только те участки, в которых исходные слои не пересекаются. Аттрибутивная таблица составляется из полей обоих слоев.
        \item Отсечение. Совмещает слои таким образом, что в выходном слое содержатся только те участки, которые пересекаются со слоем отсечения.
        \item Объединение. строится слой из пересекающихся объектов указанных слоев, атрибуты объектов объединяются в новой таблице.
        \item Объединение по признаку. Объекты с одинаковыми атрибутами объединяются в один объект.
        \item Удалить осколочные полигоны. Объединяет выделенные объекты с соседним полигоном, площадь или длина общей границы которого наибольшая.
    \end{itemize}
\end{frame}

\begin{frame}[allowframebreaks]
    \frametitle{Обработка геометрии}
    \begin{itemize}
        \item Проверка геометрии. Проверяет полигоны на наличие пересечений, «островов» и неправильного порядка нумерации узлов. (Вместо этого инструмента лучше использовать <<Модули/Проверка топологии>>)
        \item Экспортировать/добавить поле геометрии. Добавляет к слою поле(я) с информацией о геометрии: (XCOORD, YCOORD) для точечного слоя, (LENGTH) для линейного и (AREA, PERIMETER) для полигонального.
        \item Центроиды полигонов. Вычисляет истинные центроиды для каждого полигона исходного полигонального слоя.
        \item Триангуляция Делоне. Рассчитывает и строит (как полигональный шейп-файл) триангуляцию Делоне для исходного точечного слоя.
        \item Полигоны Вороного. Рассчитывает и строит полигоны Вороного для исходного точечного слоя.
        \item Упростить геометрию. Упрощает линии или полигоны при помощи модифицированного алгоритма Дугласа-Пойкера.
        \item Добавить вершины. Добавляет дополнительные вершины к объектам линейного или полиногнального слоя.
        \item Разбить составную геометрию. Преобразует составные объекты (мульти-полигоны или мульти-полилинии) в несколько простых объектов (полигонов или полилиний).
        \item Объединить геометрию в составную. Объединяет несколько простых объектов в один составной на основе значения указанного поля.
        \item Преобразовать полигоны в линии.
        \item Преобразовать линии в полигоны.
        \item Извлечение узлов. Извлекает узлы из линий или полигонов, создавая точечный шейп-файл.
    \end{itemize}

\end{frame}

\begin{frame}
    \frametitle{Управление данными}
    \begin{itemize}
        \item Задать текущую проекцию.
        \item Объединение атрибутов по районам. Присоединяет дополнительные атрибуты к векторному слою на основе пространственного взаимного расположения. Атрибуты из одного векторного слоя присоединяются к атрибутивной таблице другого векторного слоя и экспортируются в шейп-файл.
        \item Разбить векторный слой. Деление слоя на несколько отдельных слоев по указанному атрибуту.
        \item Объединение shape-файлов. Объединяет слои одинакового типа (точки, линии, полигоны) в один слой.
        \item Создать пространственный индекс.
    \end{itemize}
\end{frame}

