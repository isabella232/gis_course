

\begin{frame}[allowframebreaks]
    \frametitle{Источники информации}
    \begin{enumerate}
        \item Сайт специалистов в области         ГИС и ДЗЗ: \url{http://gis-lab.info/}
        \item Сайт открытой ГИС QGIS: \url{http://www.qgis.org/}
        \item QGIS <<русской сборки>> \url{http://nextgis.ru/nextgis-qgis/}
        \item Онлайн-документация QGIS: \url{http://www.qgis.org/ru/site/index.html}
        \item GDAL/OGR - библиотеки обработки         растровых и векторных геоданных:        \url{http://gdal.org/index_ru.html}
        \item PROJ.4: библиотека для        выполнения преобразований систем         координат: \url{http://trac.osgeo.org/proj}
        \item База данных систем координат     European Petroleum Survey Group (EPSG):        \url{http://www.epsg.org}
        \item База с описанием различных         систем координат и проекций:    \url{http://spatialreference.org}
        \item Сайт кураторов открытого ПО    ГИС: \url{http://www.osgeo.org}
        \item Сайт OpenStreetMap:  \url{http://www.openstreetmap.org}
        \item Сайт космической программы Landsat: \url{http://landsat.gsfc.nasa.gov}
        \item Сайт космической программы MODIS: \url{http://modis.gsfc.nasa.gov}
        \item Сайт геологической службы США: \url{http://www.usgs.gov}
    \end{enumerate}
\end{frame}

\begin{frame}
    \frametitle{Обзор обучающей литературы по QGIS}
    \begin{itemize}
        \item Плавное введение в ГИС: \url{http://gis-lab.info/qa/gentle-intro-gis.html}.
        \item Руководство пользователя QGIS:
            \begin{itemize}
                \item актуальная документация по последнюю версию QGIS \url{http://www.qgis.org/ru/docs/user_manual/index.html};
                \item документация в pdf формате \url{http://download.osgeo.org/qgis/doc/manual/}.
            \end{itemize}
    \end{itemize}
\end{frame}


